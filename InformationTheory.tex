\documentclass[a4paper,14pt]{extreport}
\usepackage[utf8]{inputenc}
\usepackage[english,ukrainian]{babel}
\usepackage{tempora}
\usepackage{fancyhdr} 
\usepackage{titlesec}
\usepackage{mathtools}
\usepackage{starfont}
\usepackage{longtable}
\usepackage{graphics}
\usepackage{listings}
\usepackage{caption}
\usepackage{float}

\newenvironment{metaverbatim}{\verbatim}{\endverbatim}
\DeclarePairedDelimiter{\ceil}{\lceil}{\rceil}
\usepackage{{booktabs}}
\begin{document}
\begin{titlepage}
	\begin{center}
		\large
		МІНІСТЕРСТВО ОСВІТИ І НАУКИ УКРАЇНИ
		
		\vspace{0.5cm}
		
		СУМСЬКИЙ ДЕРЖАВНИЙ УНІВЕРСИТЕТ\\
		
		\vspace{0.25cm}
		
		Кафедра електроніки і комп'ютерної техніки
		\vspace{2.5cm}
		\vfill
		
		\textsc{КОНТРОЛЬНА РОБОТА}\\[2mm]
		
		з дисципліни \\«Теорія інформації та кодування»
		\vfill
		
	
		
	\end{center}
	\vfill
	
	\newlength{\ML}
	\settowidth{\ML}{}
	Виконав аспірант групи Аз-26/КН Хитров О.Б.
	\bigskip
	

	Перевірив О. Б. Бережна
	\vfill
	
	\vspace{5cm}
	
	\begin{center}
		Суми\\ 2023 р.
	\end{center}
	
\end{titlepage}
	
	\section{Кодування}
	\subsection{Завадостійкі коди}
	Завадостійкі коди - одне з найбільш ефективних засобів забезпечення високої достовірності передачі дискреетної інформаіції.

Вхідний текст

\subsubsection{Вхідний текст}


 \begin{figure}[h]
	\centering
	\begin{minipage}{0.8\textwidth}  % Adjust the width as needed
	4 Веста (символ: \Vesta) — астероїд, один із найбільших об'єктів у поясі астероїдів (середній діаметр 525 км). Відкритий 29 березня 1807 року німецьким астроном Генріхом Ольберсом і названий на честь Вести, богині дому та вогнища в римській міфології. Вважається, що Веста є другим об'єктом у поясі астероїдів за масою та за об'ємом після карликової планети Церери, хоч об'єм Вести (у межах похибки вимірювань) дорівнює об'єму Паллади. На Весту припадає близько 9 \% маси поясу астероїдів. Для наземного спостерігача Веста є найяскравішим астероїдом. Її яскравість може досягати 5,1 зоряної величини, і в такі моменти її, хоч і слабко, видно неозброєним оком. 1,2 мільярда років тому Веста зазнала зіткнень з іншими об'єктами, що спричинили формування численних уламків та двох величезних кратерів, які займають більшу частину південної півкулі астероїда. Найбільше інформації про поверхню Вести відомо завдяки американському космічному апарату «Dawn», який дослідив астероїд в 2011—2012 роках
	\end{minipage}
	\caption*{Вхідний текст}
	\label{fig:sample-text}
\end{figure}


Потужність множини $A$ дорівніює кількості символів $a_i$ первинного алфавіту $12$. 

Переведемо вхідний текст у  верхній регістр, та відфільтруємо символи, що не належать до первинного алфавіту


 \begin{figure}[h]
	\centering
	\begin{minipage}{0.8\textwidth}  % Adjust the width as needed
	ЕА МЛ   АЕР,   АЛ  У  АЕР ЕРЕ АМЕР  М. Р  ЕРЕ  РУ МЕЦМ АРМ ГЕРМ ЛЕРМ  АА А Е Е, Г МУ А ГА  РМ МЛГ. АЖА,  ЕА  РУГМ М У  АЕР А МА А А ММ Л АРЛ ЛАЕ ЦЕРЕР,  М Е У МЕЖА  МРА Р МУ АЛЛА. А ЕУ РАА Л   МА У АЕР. Л АЕМГ ЕРГАА ЕА  АРАМ АЕРМ.  РА МЖЕ ГА , Р ЕЛ,   А ММЕ ,   ЛА,  ЕРМ М. , МЛРА Р МУ ЕА ААЛА Е  М АМ,  РЛ РМУА ЛЕ УЛАМ А  ЕЛЕ РАЕР,  АМА ЛУ АУ Е УЛ АЕРА. АЛЕ РМАЦ Р ЕР Е М А АМЕРАМУ ММУ ААРАУ ,  Л АЕР   РА.
	\end{minipage}
	
	\caption*{Вхідне джерело інформації} % Use \caption* to omit the label
	\label{fig:filtered-text}
\end{figure}



 
 \subsubsection{Опис джерела інформації}
 
 \begin{table}[H]
 	\caption{Ймовірністі $P(a_i)$ появи симмолів первинного алфавіту}
 	\centering
 	\resizebox{\columnwidth}{!}{%
 	\begin{tabular}{lrrrrrrrrrrrr}
\toprule
 &   & , & . & А & Г & Л & Ж & Е & У & М & Ц & Р \\
\midrule
Count & 143 & 12 & 8 & 66 & 8 & 23 & 3 & 43 & 18 & 42 & 3 & 40 \\
\bottomrule
\end{tabular}

 	}
 \end{table}
  

Ймовірністі $P(a_i)$ появи симмолів первинного алфавіту визначаються за формулою
$$ P(a_i) = \frac{N(a_i)}{N}$$
де $N(a_i)$ - кількість появ символу $a_i$ у профільтрованому тексті, $N$ - кількість всіх символів у профільтрованому тексті

\begin{table}[H]
	\caption{Ймовірністі $P(a_i)$ появи симмолів первинного алфавіту}
	\centering
	\resizebox{\columnwidth}{!}{%
		\begin{tabular}{lrrrrrrrrrrrr}
\toprule
 &   & , & . & А & Г & Л & Ж & Е & У & М & Ц & Р \\
\midrule
Probability & 0.349633 & 0.029340 & 0.019560 & 0.161369 & 0.019560 & 0.056235 & 0.007335 & 0.105134 & 0.044010 & 0.102689 & 0.007335 & 0.097800 \\
\bottomrule
\end{tabular}

	}
\end{table}

 

 
Сума всіх ймовірностей $P(a_i)$ повинна дорівнювати одиниці, тобто
 
 $$ \sum_{i=1}^{|A|} P(a_i) = 1 $$

\subsubsection[Кодування]{Кодування}

Кількість бітів, необхідних для представлення числа $m_1$ у двійковій системі числення визначається як найменше ціле число, яке більше або рівне логарифму $m_1$ за основою $2$.
Визначимо кількість інформаційних розрядів $n_i $ за формулою 
$$n_i =\ceil[\big]{log_2 m_1} $$
$$n_i =\ceil[\big]{log_2 12}   =\ceil[\big]{3.584962500721156} = 4  $$
\begin{table}[H]
	\caption{Кодові комбінації $|B|$}
	\centering
	\begin{tabular}{lrrrr}
\toprule
 & 0 & 1 & 2 & 3 \\
\midrule
0 & 0 & 0 & 0 & 0 \\
1 & 0 & 0 & 0 & 1 \\
2 & 0 & 0 & 1 & 0 \\
3 & 0 & 0 & 1 & 1 \\
4 & 0 & 1 & 0 & 0 \\
5 & 0 & 1 & 0 & 1 \\
6 & 0 & 1 & 1 & 0 \\
7 & 0 & 1 & 1 & 1 \\
8 & 1 & 0 & 0 & 0 \\
9 & 1 & 0 & 0 & 1 \\
10 & 1 & 0 & 1 & 0 \\
11 & 1 & 0 & 1 & 1 \\
12 & 1 & 1 & 0 & 0 \\
13 & 1 & 1 & 0 & 1 \\
14 & 1 & 1 & 1 & 0 \\
15 & 1 & 1 & 1 & 1 \\
\bottomrule
\end{tabular}

\end{table}
Оскільки $|A| < |B|$ , то потрібно обрати 12 кодових комбінацій з шістнадцяти
можливих, які забезпечать більший рівень завадостійкості.

Для відбору набійльш завадостійких комбінацій використовується 
Відстань Геммінга (англ. Hamming distance)  — число позицій, у яких відповідні цифри двох двійкових слів однакової довжини різні[1]. У загальнішому випадку відстань Геммінга застосовується для рядків однакової довжини будь-яких абеток, що складаються з $q$ символів, і служить метрикою відмінності (функцією, що визначає відстань в метричному просторі) об'єктів однакової вимірності. 

\begin{table}[H]
	\caption{Матриця кодових відстаней $d_{ij}$}
	\centering
		\resizebox{\columnwidth}{!}{%
		\begin{tabular}{lrrrrrrrrrrrrrrrr}
\toprule
 & 0000 & 0001 & 0010 & 0011 & 0100 & 0101 & 0110 & 0111 & 1000 & 1001 & 1010 & 1011 & 1100 & 1101 & 1110 & 1111 \\
\midrule
0000 & 0 & 1 & 1 & 2 & 1 & 2 & 2 & 3 & 1 & 2 & 2 & 3 & 2 & 3 & 3 & 4 \\
0001 & 1 & 0 & 2 & 1 & 2 & 1 & 3 & 2 & 2 & 1 & 3 & 2 & 3 & 2 & 4 & 3 \\
0010 & 1 & 2 & 0 & 1 & 2 & 3 & 1 & 2 & 2 & 3 & 1 & 2 & 3 & 4 & 2 & 3 \\
0011 & 2 & 1 & 1 & 0 & 3 & 2 & 2 & 1 & 3 & 2 & 2 & 1 & 4 & 3 & 3 & 2 \\
0100 & 1 & 2 & 2 & 3 & 0 & 1 & 1 & 2 & 2 & 3 & 3 & 4 & 1 & 2 & 2 & 3 \\
0101 & 2 & 1 & 3 & 2 & 1 & 0 & 2 & 1 & 3 & 2 & 4 & 3 & 2 & 1 & 3 & 2 \\
0110 & 2 & 3 & 1 & 2 & 1 & 2 & 0 & 1 & 3 & 4 & 2 & 3 & 2 & 3 & 1 & 2 \\
0111 & 3 & 2 & 2 & 1 & 2 & 1 & 1 & 0 & 4 & 3 & 3 & 2 & 3 & 2 & 2 & 1 \\
1000 & 1 & 2 & 2 & 3 & 2 & 3 & 3 & 4 & 0 & 1 & 1 & 2 & 1 & 2 & 2 & 3 \\
1001 & 2 & 1 & 3 & 2 & 3 & 2 & 4 & 3 & 1 & 0 & 2 & 1 & 2 & 1 & 3 & 2 \\
1010 & 2 & 3 & 1 & 2 & 3 & 4 & 2 & 3 & 1 & 2 & 0 & 1 & 2 & 3 & 1 & 2 \\
1011 & 3 & 2 & 2 & 1 & 4 & 3 & 3 & 2 & 2 & 1 & 1 & 0 & 3 & 2 & 2 & 1 \\
1100 & 2 & 3 & 3 & 4 & 1 & 2 & 2 & 3 & 1 & 2 & 2 & 3 & 0 & 1 & 1 & 2 \\
1101 & 3 & 2 & 4 & 3 & 2 & 1 & 3 & 2 & 2 & 1 & 3 & 2 & 1 & 0 & 2 & 1 \\
1110 & 3 & 4 & 2 & 3 & 2 & 3 & 1 & 2 & 2 & 3 & 1 & 2 & 1 & 2 & 0 & 1 \\
1111 & 4 & 3 & 3 & 2 & 3 & 2 & 2 & 1 & 3 & 2 & 2 & 1 & 2 & 1 & 1 & 0 \\
\bottomrule
\end{tabular}

	}
\end{table}
\begin{table}[H]
	\caption{Аналіз кодових відстаней $d_{ij}$}
	\centering
	\begin{tabular}{lrrrrr}
\toprule
 & 0 & 1 & 2 & 3 & 4 \\
\midrule
0000 & 1 & 4 & 6 & 4 & 1 \\
0001 & 1 & 4 & 6 & 4 & 1 \\
0010 & 1 & 4 & 6 & 4 & 1 \\
0011 & 1 & 4 & 6 & 4 & 1 \\
0100 & 1 & 4 & 6 & 4 & 1 \\
0101 & 1 & 4 & 6 & 4 & 1 \\
0110 & 1 & 4 & 6 & 4 & 1 \\
0111 & 1 & 4 & 6 & 4 & 1 \\
1000 & 1 & 4 & 6 & 4 & 1 \\
1001 & 1 & 4 & 6 & 4 & 1 \\
1010 & 1 & 4 & 6 & 4 & 1 \\
1011 & 1 & 4 & 6 & 4 & 1 \\
1100 & 1 & 4 & 6 & 4 & 1 \\
1101 & 1 & 4 & 6 & 4 & 1 \\
1110 & 1 & 4 & 6 & 4 & 1 \\
1111 & 1 & 4 & 6 & 4 & 1 \\
\bottomrule
\end{tabular}

	\label{table:codewords-distance-analysis}
\end{table}
Проведемо аналіз матриці кодових відстаней та покажемо результат цього
аналізу в додаткових в таблиці \ref{table:codewords-distance-analysis}.

На підставі аналізу цієї матриці вибираються кодові комбінації в кількості $|A|$, які
доцільно використовувати для завадостійкого кодування
інформації, та записується кодове відображення.

Аналіз матриці кодових відстаней показав, що кодові комбінації мають однаковий рівень завадостійкості.

У телекомунікаціях код Бергера — це односпрямований код виявлення помилок, названий на честь його винахідника Дж. М. Бергера. Коди Бергера можуть виявляти всі односпрямовані помилки. Односпрямовані помилки - це помилки, які перетворюють лише одиниці на нулі або лише нулі на одиниці, наприклад, в асиметричних каналах. Контрольні біти кодів Бергера обчислюються шляхом підрахунку всіх нулів в інформаційному слові та вираження цього числа в натуральній двійковій системі. Якщо інформаційне слово складається з $n$ бітів, тоді для коду Бергера потрібно $$k = [ log_2 ( n + 1 )]$$ «перевірочнх біти», що додає Код Бергера довжини $k+n$. (Іншими словами,  $k$ контрольних бітів достатньо для перевірки до

% $$n = 2 k − 1 n=2^{k}-1$$ 

інформаційних бітів). Коди Бергера можуть виявляти будь-яку кількість помилок перевертання бітів один до нуля, за умови, що в одному кодовому слові не було помилок нуль до одного. Подібним чином коди Бергера можуть виявляти будь-яку кількість помилок перетворення бітів від нуля до одного, за умови, що в одному кодовому слові не виникає помилок перетворення бітів від одного до нуля. Коди Бергера не можуть виправити жодної помилки.

Як і всі односпрямовані коди виявлення помилок, коди Бергера також можна використовувати в схемах, нечутливих до затримки.
\subsubsection{Висновок}
Більш завадостійкими є кодові комбінації, які характеризуються більшими
значеннями кодових відстаней.



\end{document}